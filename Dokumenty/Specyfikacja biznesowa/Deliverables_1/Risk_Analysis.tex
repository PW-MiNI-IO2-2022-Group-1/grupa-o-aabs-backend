\chapter{Analiza ryzyka}
%Opisy tabel na górze
%Do poprawienia
Poniższa sekcja zawiera analizę ryzyka projektu. Do opisu użyto modelu SWOT, można podzielić według dwóch kryteriów: 
\begin{enumerate}
    \item źródła:
    \begin{enumerate}
        \item wewnętrzne - dany element jest charakterystyką wewnętrzną projektu lub zespołu.
        \item zewnętrzne - dany element jest czynnikiem zewnętrznym wpływającym na projekt lub zespół.
    \end{enumerate}
    \item skutki:
    \begin{enumerate}
        \item pomocne - dany element ma pozytywny wpływ na wykonanie projektu lub na pracę członków zespołu.
        \item szkodliwe - dany element ma negatywny wpływ na wykonanie projektu lub na pracę członków zespołu.
    \end{enumerate}
\end{enumerate}
a następnie rozdzielić je do 4 kategorii: 
\begin{enumerate}
    \item "S" (ang. strengths) czyli mocne strony - wszystkie elementy projektu lub zespołu, które wpływają korzystnie na pracę.
    \item "W" (ang. weaknesses) czyli słabe strony - wszystkie elementy projektu lub zespołu, które wpływają niekorzystnie na pracę.
    \item "O" (ang. opportunities) czyli szanse - wszystkie elementy zewnętrzne, które mogą być pomocne przy realizacji projektu.
    \item "T" (ang. threats) czyli zagrożenia - wszystkie czynniki zewnętrzne, które mogą być przeszkodą w realizacji projektu.
\end{enumerate}
Każde wydarzenie ma również przypisane prawdopodobieństwo wystąpienia na pięciostopniowej skali: Praktycznie niemożliwe -- Mało prawdopodobne -- Prawdopodobne -- Bardzo prawdopodobne -- Prawie pewne.
\begin{figure}[H]
    \centering
        \caption{Diagram analizy SWOT}
        \begin{tabular}{ |c|p{4.5cm}|p{4.5cm}|  }
            \hline
            & Pomocne & Szkodliwe\\
            \hline
            Wewnętrzne & (a) Znajomość tematu gier komputerowych & (a) Nieznajomość środowiska pracy \\
            & (b) Zgrany zespół & (b) Różnorodność potrzebnej wiedzy \\
            \hline
            Zewnętrzne & (a) Popularność tematu tworzenia gier & (a) Możliwa zmienność nowego oprogramowania\\
            & (b) Szeroko wykorzystywana tematyka pracy & (b) Opóźnienia związane z~życiem prywatnym\\
            \hline
        \end{tabular}
         \index{Diagram analizy SWOT}
\end{figure}

\begin{enumerate}
    \item Mocne strony
        \begin{enumerate}
            \item Znajomość tematu gier komputerowych 
            
            Prawdopodobieństwo: Prawie pewne
            
            Opis: Szeroka wiedza i~doświadczenie w~temacie gier komputerowych pozwala nam na jasne określenie wymagań, przewidzenie potencjalnych trudności oraz funkcjonalności, które powinny być uwzględnione.
            \item Zgrany zespół 
            
            Prawdopodobieństwo: Prawie pewne
            
            Opis: Doświadczenie współpracy przy innych projektach pozwala nam odpowiednio dopasować zadania do naszych silnych stron i zmniejszyć wpływ słabych na jakość oraz wydajność pracy.
        \end{enumerate}
        
    \item Słabe strony
        \begin{enumerate}
            \item Nieznajomość środowiska pracy
            
            Prawdopodobieństwo: Prawdopodobne
            
            Opis:  Brak wcześniejszej znajomości środowiska pracy może prowadzić do wydłużenia poszczególnych etapów związanych z pracą z~nim.
            \item Różnorodność potrzebnej wiedzy 
            
            Prawdopodobieństwo: Prawdopodobne
            
            Opis:  Zakres tematyczny potrzebny do zaimplementowania aplikacji wymaga styczności z obszernymi tematami, takimi jak tworzenie gier i generowanie proceduralne, co może pociągnąć za sobą potrzebę nauczenia się korzystania z wielu narzędzi, wydłużając czas pracy.
        \end{enumerate}
        
    \item Szanse
        \begin{enumerate}
            \item Popularność tematu tworzenia gier
            
            Prawdopodobieństwo: Bardzo prawdopodobne
            
            Opis:  Tworzenie gier jest obecnie popularnym tematem, co przekłada się na dostępność dokumentacji, poradników oraz różnorodnych pakietów z~zasobami dla popularnych silników tworzenia gier.
            \item Szeroko wykorzystywana tematyka pracy 
            
            Prawdopodobieństwo: Prawdopodobne
            
            Opis: Technologia generowania proceduralnego jest powszechnie wykorzystywanym mechanizmem, co ułatwia znalezienie materiałów.
        \end{enumerate}
        
    \item Zagrożenia
        \begin{enumerate}
            \item Możliwa zmienność silników tworzenia gier
            
            Prawdopodobieństwo: Mało prawdopodobne
            
            Opis: Potencjalne prace twórców silników tworzenia gier nad nimi mogą prowadzić do potrzeby wprowadzania zmian.
            \item Opóźnienia związane z~życiem prywatnym  
            
            Prawdopodobieństwo: Prawdopodobne
            
            Opis: Choroba bądź większe wymagania czasowe w~pracy mogą spowodować czasowe spowolnienie bądź wstrzymanie pracy nad projektem.
        \end{enumerate}
\end{enumerate}
