\section{Wymagania niefunkcjonalne}
Sekcja zawiera informacje na temat wymagań niefunkcjonalnych z modelu FURPS, gdzie 
\begin{enumerate}
    \item "F" (ang. functionality) - oznacza wymagania funkcjonalne.
    \item "U" (ang. utility) - oznacza wymagania dotyczące użyteczności.
    \item "R" (ang. reliability) - oznacza wymagania dotyczące niezawodności.
    \item "P" (ang. performance) - oznacza wymagania dotyczące wydajności.
    \item "S" (ang. supportability) - oznacza wymagania dotyczące utrzymania.
\end{enumerate} 
Poniżej znajdują się informacje na temat tych wymagań z wyłączeniem wymagań funkcjonalnych, które zostały opisane w poprzedniej sekcji.

\begin{enumerate}
    \item Użyteczność
        \begin{enumerate}
            \item Gra działa na systemie Windows 10 64-bitowym z procesorem Quad-core Intel lub AMD, 2.5 GHz lub szybszym, kartą graficzną kompatybilną z DirectX 11 lub 12 oraz  pamięcią RAM co najmniej 8GB.
            \item Bliskie otoczenie gracza jest wyraźne i rozróżnialne na ekranie o rozdzielczości $800\times 600$.
            \item Wszystkie opcje gry mieszczą się na ekranie o rozdzielczości $800\times 600$, a minimalny rozmiar użytej czcionki wynosi 12 pt.
            \item Zmniejszenie rozdzielczości gry do $800\times 600$ nie wpływa na czytelność napisów lub grywalność aplikacji.
        \end{enumerate}
    
    \item Niezawodność
        \begin{enumerate}
            \item Gra uruchomiona po awarii umożliwia wczytanie ostatniego zapisu.
            \item Gra posiada system automatycznego zapisu.
            \item Awaria gry w trakcie zapisu nie prowadzi do usunięcia poprzedniego zapisu.
        \end{enumerate}
        
    \item Wydajność
        \begin{enumerate}
            \item Gra działa w średnio 30 klatkach na sekundę na niskich ustawieniach graficznych (brak cieni, niska jakość oświetlenia, mała odległość renderowania) na komputerze spełniającym minimalne wymagania sprzętowe.
            \item Czas trwania pojedynczego zapisu nie przekracza jednej minuty.
            \item Czas wczytania istniejącego zapisu nie przekracza dwóch minut.
        \end{enumerate}
        
    \item Utrzymanie
        \begin{enumerate}
            \item Użytkownicy będą mogli wysyłać zgłoszenia o błędach na podany adres e-mail.
            \item Wypuszczane łaty będą dostępne dla użytkowników do pobrania z repozytorium.
        \end{enumerate}
\end{enumerate}